\documentclass[10pt,letterpaper]{article}
\usepackage[margin=0.5in]{geometry}
\usepackage[utf8]{inputenc}
\usepackage[T1]{fontenc}
\usepackage[stretch=10]{microtype}
\usepackage{tgpagella}
\usepackage{enumitem}
\pagestyle{empty}
\usepackage[none]{hyphenat}
\usepackage[hidelinks]{hyperref}

%:Begin Document
\begin{document}
\begin{center}
{\huge \textbf{Vincent Kee}}

\href{http://vincentkee.wordpress.com}{vincentkee.wordpress.com}\ \ \textbullet
\ \ \href{mailto:vkee@mit.edu}{vkee@mit.edu}\ \ \textbullet
\ \  Phone: (310) 528 - 8831
\end{center}

\hrule
\vspace{-0.6em}

\subsection*{Education}
  \begin{itemize}
    \parskip=-0.1em

    \item[]
    {\href{http://mit.edu/}{\textbf{Massachusetts Institute of Technology (MIT)}} \hfill
      \textbf{Cambridge, MA}}
    \\
        {\emph{Candidate for Master of Engineering in Electrical Engineering and Computer Science}\hfill 
        \emph{June 2017}} 
        \begin{itemize}	
            \parskip=-0.1em
        \item[] Concentration: Artificial Intelligence
        \item[] Relevant Courses: \href{http://student.mit.edu/catalog/search.cgi?search=2.166&style=verbatim}{Autonomous Vehicles}, \href{http://student.mit.edu/catalog/search.cgi?search=6.832&style=verbatim}{Underactuated Robotics}, \href{http://student.mit.edu/catalog/search.cgi?search=6.834&style=verbatim}{Cognitive Robotics}
         \end{itemize}
    {\emph{Bachelor of Science in Electrical Engineering and Computer Science;} \, GPA: 4.6/5.0 \hfill
      \emph{June 2016}}
      \begin{itemize}
          \parskip=-0.1em
      \item[] Relevant Courses: \href{http://student.mit.edu/catalog/search.cgi?search=6.141&style=verbatim}{Robotics: Science and Systems}, \href{http://student.mit.edu/catalog/search.cgi?search=6.869&style=verbatim}{Computer Vision}, \href{http://student.mit.edu/catalog/search.cgi?search=6.302&style=verbatim}{Feedback Systems}, \href{http://student.mit.edu/catalog/search.cgi?search=6.006&style=verbatim}{Algorithms}
   \end{itemize}   
  \end{itemize}

\hrule
\vspace{-0.6em}

%: Work Experience
\subsection*{Work Experience}
  \begin{itemize}
    \parskip=-0.1em

%: Draper Internship
    \item[]
    {\href{http://draper.com/}{\textbf{Perception and Localization Group, Draper}} \hfill
      \textbf{Cambridge, MA}}
    \\
        {\emph{Draper Fellow} \hfill \emph{June 2016 - Present}}
	
	\begin{itemize}[label=\textbullet]
	\itemsep0em 
	\item Working on the perception pipeline to enable an autonomous robot to perform an oil change on any car
	\end{itemize} 
    {\emph{Draper Laboratory Undergraduate Research and Innovation Scholar} \hfill \emph{September 2015 - May 2016}}
	
	\begin{itemize}[label=\textbullet]
	\itemsep0em 
	\item Integrated deformation graphs into the dense visual SLAM  algorithm KinectFusion to improve scene reconstruction quality in large-scale environments using RGB-D camera depth data
	\end{itemize} 
		{\emph{Signal Processing, Algorithms, \& Software Summer Student} \hfill \emph{May 2015 - August 2015}}
	\begin{itemize}[label=\textbullet]
	\item Implemented methods to robustify the iterative closest point component of the KinectFusion algorithm 
	
\end{itemize}

%: ACL UROP
    \item[]
    {\href{http://acl.mit.edu/}{\textbf{Aerospace Controls Lab, MIT Laboratory for Information and Decision Systems (LIDS)}} \hfill
      \textbf{Cambridge, MA}}
    \\
    {\emph{Undergraduate Researcher} \hfill \emph{January - May 2015}}
	
	\begin{itemize}[label=\textbullet]
	\itemsep0em 
	\item Worked on path planning algorithms for navigating a car through a pedestrian rich environment
	
\end{itemize}

%: Aurora Flight Sciences Internship
    \item[]
    {\href{http://aurora.aero}{\textbf{Aerial Robotics Group, Aurora Flight Sciences}} \hfill
      \textbf{Manassas, VA}}
    \\
    {\emph{Electrical Engineering and Computer Science Intern} \hfill \emph{June - August 2014}}
	
	\begin{itemize}[label=\textbullet]
	\itemsep0em
	\item Developed path planning algorithms for scanning areas with UAVs along with workflow and software application
	\item Conducted two demos of mission planning tool and presented work to audience of over 50 company engineers

\end{itemize}

%: RoboMods MISTI Singapore
    \item[]
    {\href{http://www.sutd.edu.sg/idc.aspx}{\textbf{MIT-SUTD International Design Centre, Singapore University of Technology and Design (SUTD)}} \hfill
      \textbf{Singapore}}
    \\
    {\emph{Undergraduate Researcher} \hfill \emph{July - August 2013}}
	
	\begin{itemize}[label=\textbullet]
	\itemsep0em
	\item {\href{http://vincentkee.wordpress.com/tetromino/}{Designed and fabricated a nested reconfiguration modular robotics system capable of rearranging its own modules (intra-reconfiguration) and combining with other systems to form more complex systems (inter-reconfiguration)}}
	\item {\href{http://www.youtube.com/watch?v=YFhAlsQ3uYQ}{Demonstrated intra-reconfiguration into all seven one-sided tetrominoes with the hinged tetromino prototype}}

\end{itemize}


%: BU Research Internship
%    \item[]
%    {\href{http://nl.bu.edu/}{\textbf{CELEST Neuromorphics Lab, Boston University}} \hfill
%      \textbf{Boston, MA}}
%    \\
%    {\emph{Research Intern} \hfill \emph{July - August 2011}}
%	
%	\begin{itemize}[label=\textbullet]
%	\itemsep0em
%	\item Developed optic-flow detecting filters and navigational algorithms for autonomous robots
%	\item Presented \href{http://vincentkee.files.wordpress.com/2011/09/optic-flow-based-navigation-using-correlation-techniques.png}{poster} to BU Integrated Circuits and Systems group and \textasciitilde 50 people at a BU public poster session
%	\end{itemize}
\end{itemize}

\hrule
\vspace{-0.6em}

\subsection*{Projects}
  \begin{itemize}
    \parskip=-0.1em
    
%: 6.869
    \item[]
    {\href{http://6.869.csail.mit.edu/}{\textbf{Advances in Computer Vision Final Project - Scene Recognition}} \hfill
      \textbf{November - December 2015}}

    \begin{itemize}[label=\textbullet]
      \itemsep0em
      \item Designed deep convolutional neural networks for scene recognition and trained them with the Mini Places Challenge dataset, a subset of the \href{http://places2.csail.mit.edu/}{Places2 Challenge dataset} as part of a three person team
      \end{itemize}    
    
%: 6.141
    \item[]
    {\href{http://courses.csail.mit.edu/rss/}{\textbf{Robotics: Science and Systems Final Project - A Fully Autonomous Localizing Mobile Robot}} \hfill
      \textbf{February - May 2015}}

    \begin{itemize}[label=\textbullet]
      \itemsep0em
      \item Implemented a particle filter utilizing RGB-D camera sensor data and an adaptive RRT-based motion planning algorithm for a fully autonomous localizing mobile robot as part of a four person team
      \end{itemize}    
\end{itemize}

\hrule
\vspace{-0.6em}

%: Technical Skills
\subsection*{Technical Skills}
\begin{itemize}
	\item[] C++, Java, Python, MATLAB, Robot Operating System (ROS), JavaScript
\end{itemize}

\hrule
\vspace{-0.6em}

\subsection*{Awards}
  \begin{itemize}
    \parskip=-0.1em
    
%: MASLAB
    \item[]
    {\href{http://maslab.mit.edu/2014/site/index.html}{\textbf{2nd Place in MIT Mobile Autonomous Systems Laboratory Robotics Competition}} \hfill
      \textbf{January 2014}}

    \begin{itemize}[label=\textbullet]
      \itemsep0em
      \item Designed, manufactured, and programmed a fully autonomous robot utilizing computer vision and ultrasonic sensor feedback for navigation in one month as part of a five person team
      \end{itemize}    
\end{itemize}

\hrule
\vspace{-0.6em}

\subsection*{Selected Publications}
    \parskip=-0.1em
{\href{http://ieeexplore.ieee.org/xpls/abs_all.jsp?arnumber=6878302}{Kee, V., Rojas, N., Elara, M. R., \& Sosa, R. (2014, July). Hinged-Tetro: A self-reconfigurable module for nested reconfiguration. In Advanced Intelligent Mechatronics (AIM), 2014 IEEE/ASME International Conference on (pp. 1539-1546). IEEE.}}

\end{document}